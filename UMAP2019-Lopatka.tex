% Set options for wide aspect ratio and beamer document class
\documentclass[aspectratio=169]{beamer}

% Package imports
\usepackage{fontspec}
\usepackage{epigraph}
\usepackage{xcolor}
\usepackage[percent]{overpic}

% custom options to make epigraph look good on a beamer slide
\renewcommand{\epigraphrule}{0pt}
\setlength{\epigraphwidth}{.9\textwidth}
\renewcommand{\textflush}{flushepinormal}

\newcommand\blfootnote[1]{%
  \begingroup
  \renewcommand\thefootnote{}\footnote{#1}%
  \addtocounter{footnote}{-1}%
  \endgroup
}

% default document font specifications
\setmainfont{FiraSans-Book}
\setsansfont{FiraSans-Book}
\definecolor{textcolour}{rgb}{255,255,255}

% Definign the Mozilla colour palette for presentations 
\definecolor{moz_dark_green}{RGB}{77,78,83}
\definecolor{moz_light_green}{RGB}{208,211,212}
\definecolor{moz_light_blue}{RGB}{0,150,221}
\definecolor{moz_dark_blue}{RGB}{0,33,71}
\definecolor{moz_accent_green}{RGB}{111,190,74}
\definecolor{moz_accent_yellow}{RGB}{255,203,0}
\definecolor{moz_accent_orange}{RGB}{255,149,0}
\definecolor{moz_accent_orange2}{RGB}{230,96,0}
\definecolor{moz_accent_red}{RGB}{193,56,50}

% new counter for column enumerates
\newcounter{savedenum}
\newcommand*{\saveenum}{\setcounter{savedenum}{\theenumi}}
\newcommand*{\resume}{\setcounter{enumi}{\thesavedenum}}

% Background for all slides set here
\setbeamertemplate{background canvas}{\includegraphics [width=\paperwidth]{bg_black.png}}

\setbeamercolor{structure}{fg=textcolour}

\setbeamerfont{title}{size = \Huge}
\setbeamercolor{normal text}{fg=textcolour}

%define specific fint size interpretations for the beamer presentation mode.
\renewcommand{\tiny}{\fontsize{7pt}{8pt}\selectfont}
\renewcommand{\scriptsize}{\fontsize{9pt}{12pt}\selectfont}
\renewcommand{\footnotesize}{\fontsize{10pt}{12pt}\selectfont}
\renewcommand{\small}{\fontsize{12pt}{18pt}\selectfont}
\renewcommand{\normalsize}{\fontsize{14pt}{18pt}\selectfont}
\renewcommand{\large}{\fontsize{16pt}{24pt}\selectfont}
\renewcommand{\Large}{\fontsize{24pt}{37pt}\selectfont}
\renewcommand{\LARGE}{\fontsize{31pt}{42pt}\selectfont}
\renewcommand{\huge}{\fontsize{48pt}{54pt}\selectfont}
\renewcommand{\Huge}{\fontsize{80pt}{96pt}\selectfont}

% remove navigation symbols
\setbeamertemplate{navigation symbols}{}
 
% After building the PDF you may ensure that local system fonts are embedded in the final PDF using:
% gs -dNOPAUSE -dBATCH -sDEVICE=pdfwrite -dPDFSETTINGS=/prepress -dEmbedAllFonts=true -sOutputFile=DN2018-lopatka-portable.pdf -f DN2018-Lopatka.pdf
% This will also compress the final output size 

\setbeameroption{show notes}
%\setbeameroption{show notes on second screen=right}

\begin{document}

\begin{frame}
%
\begin{overpic}[width=0.7\textwidth]{taar.png}
\end{overpic}
%
\note{ \item{Personalised web browsing experience is hard}
\item{Especially with a rigorous and respectful privacy policy}
\item{Ultimately, many of the approaches in UMAP strive to find innovative ways to extract a meaningful signal from very noisy data.}
}
\end{frame}

{
\usebackgroundtemplate{\includegraphics[width=\paperwidth]{bg_alt_small-rlogo.png}}
\begin{frame}
\begin{overpic}[width=0.7\textwidth]{TAAR.png}
\put(85, 7){\large{Martin Lopatka}}
\put(78, 1){\small{mlopatka@mozilla.com}}
\put(57.25, -5){\small{Data Scientist/Applied Statistician}}
\end{overpic}
%
\note{
\item{My name is Martin Lopatka}
\item{I'd like to use the time we have today to go through an approach we use at Mozilla to recommend browser extensions}
\item{ give the time, I want to focus on a very brief overview, and two design choices in particular}
\item{ differential privacy for frequency based recommendations}
\item{cllr. as a way top get more utility out of general linear stacked ensembles of a particular type}
\item{While imposing some privacy guarantees at the architectural level of TAAR's implementation}
}
%
\end{frame}
}

{
\usebackgroundtemplate{\includegraphics[width=\paperwidth]{bg_alt_small-rlogo.png}}
\begin{frame}
\frametitle{signal and the noise}
extract a meaningful signal from a quantity of collected data that will be both useful and generalizable
%
\note{
speaker note
}
%
\end{frame}
}

{
\usebackgroundtemplate{\includegraphics[width=\paperwidth]{bg_alt_small-rlogo.png}}
\begin{frame}
\frametitle{title}
content
%
\note{
speaker note
}
%
\end{frame}
}


\end{document}
